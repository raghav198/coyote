\documentclass{article}

\usepackage{amsmath}
\usepackage{proof}
\usepackage{syntax}

\title{Copse++}
\author{Raghav Malik}
\date{}

\begin{document}
\maketitle
I'm {\em very} open to renaming, Copse++ was the first thing that popped into my head.
\section{Description}
Copse++ has two IR languages; I've called them IR-1 and IR-2.
% IR-1 is higher level and more natural to write programs in, whereas IR-2 corresponds to the decision tree evaluations.
IR-1 is intended to be a direct compilation target from a higher level language, and as such its grammar corresponds more closely to a natural way of representing programs.
By contrast, IR-2 is a lower level representation that directly encodes decision trees and targets the Copse backend.
As such, a compiler from IR-1 to IR-2 essentially converts programs into decision trees that can be evaluated.
What follows is now an outline of the grammars for IR-1 and IR-2, as well as the rewrite rules for compilation.
\section{IR-1 Grammar}
\begin{grammar}

    <program> ::= `inputs' <var>* `outputs' <var>* <chain>

    <chain> ::= nil \alt <op> `->' <chain> \alt <output>

    <op> ::= `compare1' <expr> <expr> <chain> <chain> \alt
    `compare2' <expr> <expr> <chain> <chain> \alt
    `assign' <var> <expr>

    <var> ::= [a-z]

    <val> ::= [0-9]+

    <term> ::= <var> \alt <val>

    <expr> ::= <term>*

    <output> ::= `output' <var>* \alt `secret-output' <var>*


\end{grammar}
A program consists of a declaration of the input and output variables, followed by a {\em continuation chain} (the linked-list equivalent of a block).
The only operations supported right now are variable assignment, and comparison using the less-than operator; all expressions are sums of indvidual terms (e.g. $a + b + 4$).

Every program implicitly includes an instruction to output all of the output variables as the last link in the chain, so that's not specified in the grammar.

The instructions `compare1' and `compare2' are identical except in the IR-2 they produce.
\section{IR-2 Grammar}
\begin{grammar}
    <node> ::= <phi1-node> \alt <phi2-node> \alt `output' <var>* \alt `secret-output' <var>*

    <phi1-node> ::= `phi1' <expr> <expr> <node> <node>

    <phi2-node> ::= `phi2' <var>* <node> <node>
\end{grammar}
The $\phi_1$ and $\phi_2$ nodes represent the two different ways to compile a comparison (corresponding to compare1 and compare2, respectively).
$\phi_1$ is a branch in a classic decision tree, comparing two expressions and choosing one of two paths to follow depending on the comparison result.
In the world of secure computation, it can be thought of as introducing nondeterminism, causing the execution to split and follow both paths and end up in a ``superposition'' of the final states.
$\phi_2$ does the opposite of this.
A $\phi_2$ node evaluates its left subtree, the result of which is a superposition of all possible program states along with an encrypted bitvector encoding the actual path that would be taken as a result of the input values.
These are both double-blinded and sent back to the client (or one of the clients in the case of a traditional MPC setup) for measurement.
The measurement effectively ``collapses'' the program state; the collapsed state is returned to the evaluator which uses it as a secret input in evaluating the right subtree of the $\phi_2$ node.

Intermittently collapsing the states by inserting $\phi_2$ nodes into the IR reduces the total width and number of branches in the final set of decision trees at the cost of additional rounds of communication; an optimal program balances between these two factors.
\subsection{Double-blinded state collapsing}
When there are multiple parties involved in the MPC setup, it may be desirable to prevent any of them from learning intermediate values in the program execution.
The double-blinded state collapsing protocol accomplishes this.

The nondeterministic program state is a set of homomorphically encrypted vectors, where each slot of a vector holds one possible value for one variable in the program state.
We assume (as is common in MPC scenarios) that the program itself (including structure and constant values) is public knowledge; this means that the evaluator can easily construct these vectors, even without knowing what values are in them.
The output that Copse produces when evaluating a traditional vectorized and encrypted decision tree is simply a one-hot bitvector, where each slot corresponds to a leaf node of the tree; this bitvector encodes which label was selected, but not what the label was.
This bitvector is homomorphically encrypted, so it is unreadable by the evaluator but can be easily opened by the client.
The bitvector can be used to collapse the nondeterminism into a single true program state; however, simply doing this violates the double-blind.
Instead, the evaluator first constructs a random permutation matrix.
The matrix is applied to both the bitvector and every vector of the nondeterministic state; thus, they are still in correspondence with each other, but both have been shuffled the same way.
The evaluator then randomly generates an AES key and homomorphically applies AES encryption to each element of each vector in the nondeterministic program state, making it unreadable to the client.
These are then sent to the client, who homomorphically decrypts both to get a plaintext bitvector and a set of vectors of AES-encrypted values, neither of which reveal anything about the actual intermediate value.
However, they can use the plaintext bitvector to select the AES-encrypted value for each variable, effectively collapsing the superposition of states into a single one.
This is then homomorphically encrypted and sent back to the evaluator, who then uses its key to homomorphically remove the AES encryption, resulting in a single homomorphically encrypted value for each intermediate variable at the time the phi-node was encountered.
\section{Rewrite Rules for Compilation}
The following inference rules define the rewrite relation $a \implies b$ from IR-1 to IR-2 (read as ``a rewrites to b'').
The operator $V(c_1,\dots,c_n)$ denotes the set of all variables defined in continuation chains $c_1,\dots,c_n$.
The syntax $c[v:=e]$ means ``replace all instances of $v$ in $c$ with $e$''
\begin{itemize}
    \item[] \infer{\texttt{compare1}~e_1~e_2~c_1~c_2~\texttt{->}~c_3\implies \texttt{phi1}~e_1~e_2~(c_1'~\texttt{->}c_3')~(c_2'\texttt{->}c_3')}{c_1\implies c_1' & c_2\implies c_2'& c_3\implies c_3'}
    \item[] \infer{c_0~\texttt{->}~(\texttt{compare2}~e_1~e_2~c_1~c_2~\texttt{->}~c_3)\implies \texttt{phi2}~V(c_0, c_1, c_2)~c_0'~c_3'}{c_0~\texttt{->}~(\texttt{compare1}~e_1~e_1~c_1~c_2)~\texttt{->}~\texttt{secret-output}~V(c_0, c_1, c_2)\implies c_0' & c_3\implies c_3'}
    \item[] \infer{\texttt{assign}~v~e~\texttt{->}~c\implies c'[v:=e]}{c\implies c'}
\end{itemize}
Forms like {\em output}, {\em secret-output}, and expressions and variables are mapped from IR-1 to IR-2 unchanged.
\end{document}