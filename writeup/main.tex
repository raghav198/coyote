%  \documentclass[acmsmall,10pt,review,anonymous]{acmart}
\documentclass[sigconf,screen]{acmart}
% \settopmatter{printfolios=true,printccs=false,printacmref=false}
% \documentclass[acmsmall,anonymous,10pt]{acmart}

\newcommand{\asplossubmissionnumber}{399}

\usepackage[normalem]{ulem}

\usepackage[ruled,vlined]{algorithm2e}
% \usepackage{algpseudocode}
% \usepackage{balance}
% \usepackage{amssymb}
% \usepackage{mathrsfs}
% \usepackage{microtype}
\usepackage{listings}
\usepackage{graphicx}
\usepackage{xspace}
\usepackage{booktabs}
\usepackage{subcaption}
\usepackage{minted}
\usepackage{microtype}
\usepackage{enumitem}

\usepackage{xcolor}
% \usepackage{soul}

% \citestyle{acmauthoryear}
% Comment macros
\newcommand{\milind}[1]{\textcolor{blue}{\sl{\bf Milind:} #1}}
\newcommand{\raghav}[1]{\textcolor{orange}{\sl{\bf Raghav:} #1}}
% \newcommand{\raghav}[1]{}
\newcommand{\kabir}[1]{\textcolor{magenta}{\sl{\bf Ben:} #1}}

\newcommand{\system}{\text{Coyote}\xspace}

\newcommand\samplefrom{\mathrel{\reflectbox{$\leadsto$}}}
% \newcommand{\Description}[1]{}
% \newcommand{\citet}[1]{\cite{#1}}

\title{Coyote: A Compiler for Vectorizing Encrypted Arithmetic Circuits}

\author{Raghav Malik}
\affiliation{
    \department{School of Electrical and Computer Engineering}
    \institution{Purdue University}
    \city{West Lafayette}
    \state{IN}
    \country{USA}}
\email{malik22@purdue.edu}

\author{Kabir Sheth}
\affiliation{
    \department{School of Electrical and Computer Engineering}
    \institution{Purdue University}
    \city{West Lafayette}
    \state{IN}
    \country{USA}}
\email{kdsheth@purdue.edu}

\author{Milind Kulkarni}
\affiliation{
    \department{School of Electrical and Computer Engineering}
    \institution{Purdue University}
    \city{West Lafayette}
    \state{IN}
    \country{USA}}
\email{milind@purdue.edu}
%%% The following is specific to ASPLOS '23 and the paper
%%% 'Coyote: A Compiler for Vectorizing Encrypted Arithmetic Circuits'
%%% by Raghav Malik, Kabir Sheth, and Milind Kulkarni.
%%%
\setcopyright{rightsretained}
\acmPrice{}
\acmDOI{10.1145/3582016.3582057}
\acmYear{2023}
\copyrightyear{2023}
\acmSubmissionID{asplosc23main-p399-p}
\acmISBN{978-1-4503-9918-0/23/03}
\acmConference[ASPLOS '23]{Proceedings of the 28th ACM International Conference on Architectural Support for Programming Languages and Operating Systems, Volume 3}{March 25--29, 2023}{Vancouver, BC, Canada}
\acmBooktitle{Proceedings of the 28th ACM International Conference on Architectural Support for Programming Languages and Operating Systems, Volume 3 (ASPLOS '23), March 25--29, 2023, Vancouver, BC, Canada}
\received{2022-10-20}
\received[accepted]{2023-01-19}
\begin{document}
\begin{abstract}

Fully Homomorphic Encryption (FHE) is a scheme that allows a computational circuit to operate on encrypted data and produce a result that, when decrypted, yields the result of the unencrypted computation. While FHE enables privacy-preserving computation, it is extremely slow. However, the mathematical formulation of FHE supports a SIMD-like execution style, and hence recent work has turned to vectorization to recover some of the missing performance. Unfortunately, these vectorization approaches do not work well for arbitrary computations: they do not account for the high cost of {\em rotating} vector operands to allow data to be used in multiple operations. Hence, the cost of rotation can outweigh the benefits of vectorization.

This paper presents \system, a new approach to vectorizing encrypted circuits that specifically aims to optimize the use of rotations. It vectorizes entire subcircuits at once to eliminate rotations within those subcircuits. It then attempts to ``line up'' operations between subcircuits using a minimal number of rotations. \system uses a careful encoding of the {\em lane placement} problem that allows a solver to identify good rotations without having to explore an impractically large search space. This paper shows that \system is effective at vectorizing computational kernels while minimizing rotation, thus finding efficient vector schedules and smart rotation schemes to achieve substantial speedups.

\end{abstract}

\begin{CCSXML}
    <ccs2012>
       <concept>
           <concept_id>10002978.10002979</concept_id>
           <concept_desc>Security and privacy~Cryptography</concept_desc>
           <concept_significance>500</concept_significance>
           </concept>
       <concept>
           <concept_id>10011007.10011006.10011041</concept_id>
           <concept_desc>Software and its engineering~Compilers</concept_desc>
           <concept_significance>500</concept_significance>
           </concept>
       <concept>
           <concept_id>10011007.10011006.10011050.10011017</concept_id>
           <concept_desc>Software and its engineering~Domain specific languages</concept_desc>
           <concept_significance>500</concept_significance>
           </concept>
       <concept>
           <concept_id>10002978.10003022</concept_id>
           <concept_desc>Security and privacy~Software and application security</concept_desc>
           <concept_significance>300</concept_significance>
           </concept>
     </ccs2012>
\end{CCSXML}
    
\ccsdesc[500]{Security and privacy~Cryptography}
\ccsdesc[500]{Software and its engineering~Compilers}
\ccsdesc[500]{Software and its engineering~Domain specific languages}
\ccsdesc[300]{Security and privacy~Software and application security}

\keywords{Homomorphic Encryption, Arithmetic Circuits, Vectorization}

\maketitle

\section{Introduction}\label{sec:intro}
\begin{itemize}
    \item Motivation
    \item Results
    \item Contributions
    \item Layout of the paper
\end{itemize}

%% Motivation
% What's the big picture problem?
Fully Homomorphic Encryption (FHE) refers to any encryption scheme that allows for homomorphically adding and multiplying ciphertexts, so that the sum of the encryptions of two integers is an encryption of their sum, and similarly the product of the encryptions of two integers is an encryption of their product.
While FHE is an incredibly powerful technique for carrying out privacy-preserving computations on encrypted data, it has a major downside: its slow.
Homomorphic computations over ciphertexts are often orders of magnitude slower than their plaintext counterparts.
\raghav{I wonder if reviewers will read this and think ``hmm there was a PLDI paper last year that said the same thing''}
Many FHE cryptosystems support packing large numbers of ciphertexts into {\em ciphertext vectors}, essentially compensating for the inherent slowness of FHE by enabling SIMD-style computation.
To properly take advantage of ciphertext packing, we need a compiler that can vectorize arbitrary FHE programs.

% Who has tried to solve this problem before?
Vectorizing compilers for FHE are nothing new \cite{CHET, Porcupine}.
However, these are optimized for highly regular computations over packed tensors (such as neural networks in the case of CHET), and often don't take into account how expensive rotations can be in more irregular applications.
A lot of FHE computations don't work with regular data structures like tensors, which makes it much harder to take advantage of vectorization opportunities.

Other approaches to vectorizing arbitrary code that's not in a loop, such as SLP \cite{SLP}, also fail here.
SLP aggressively packs isomorphic instructions into vectors, because it assumes that shuffling vector lanes around or indexing into a vector is relatively cheap.
In FHE, however, the vectors are not physical vector registers with slots for data, but rather {\em abstract mathematical objects} which only {\em encode} several ciphertexts.
The only way to move data between vector lanes in FHE is by performing a cyclic rotation of the entire vector.
While it is theoretically possible to encode arbitrary permutations as a series of masks and rotates, realizing the shuffles incurred by SLP in this way can quickly outweigh any benefits from vectorizing in the first place.
This means that we need a new arbitrary vectorization strategy that is {\em FHE-aware}; i.e., it packs instructions without relying on regularity in the original computation, but still accounts for the high cost of data movement.


%% Contributions

%% Results

\subsection{Related Work} % limit this to basically a couple sentences, not a whole subsection
% "These are approaches people have tried that don't work"
\subsubsection{Vectorizing FHE} \raghav{CHET and Porcupine: not general}
\subsubsection{SLP} \raghav{Assumes shuffling data between lanes is cheap}
\subsection{Fully Homomorphic Encryption}
Fully Homomorphic Encryption (FHE) refers to any encryption scheme with the property that ciphertexts can be added and multiplied, and these operations commute with the encryption and decryption functions.
\subsubsection{Limitations}
\subsubsection{Arithmetic Circuits}
\subsubsection{Vectorization}
\subsection{Related Work}
\subsubsection{Vectorizing FHE Programs}
\subsection{Vectorization}
\raghav{So\dots this is weird, because I want to talk about vectorization as it pertains to FHE (i.e. ``a property of FHE schemes is they allow for packing a huge number of plaintexts into a single ciphertext'') but also give the background of vectorization terminology I'll be using (in particular the idea of lanes and how lining things up makes vectorization very different from parallelism). }
\milind{I'd put this as part of background then. You can even move 2.1.3 into a "Background on vectorization" section, and call it "Vectorization in FHE" -- that would also let you discuss why it's different than traditional SIMD vectorization.}
\system is a staging compiler for Python.
The runtime provides a \texttt{CoyoteVar} datatype which represents a symbolic encrypted value, and supports addition, subtraction, and multiplication.
Python code that manipulates these symbolic variables produces an arithmetic circuit representation of the program, which \system then compiles into a set of vector primitives that can be further lowered into C++ code that targets Microsoft's SEAL backend for BFV \raghav{(citation needed)}.
 
\subsection{Vectorization}
\system reduces the search space for vectorized programs \raghav{Does it make sense to call it ``search space''?} by first splitting the computation into multiple stages, and only allowing for vector rotations to line operands up between stages.
A stage consists of a set of independent subgraphs of the entire circuit \raghav{Do I need to explain what independent means here?}. 
The subgraphs are scheduled together, with each one being placed on its own vector lane. 
Once vector code for a stage has been generated, all of its subgraphs are removed from the circuit, and the subgraphs for the next stage are picked out.
After the entire circuit has been split up into stages, the subgraphs from each stage are assigned specific vector lanes to minimize the number of distinct rotations required to line up the outputs of one stage with the inputs of the next.
Finally, \system computes and inserts these rotation instructions between stages and emits the vector code.
\section{Design}\label{sec:design}
When vectorizing arithmetic circuits with an SLP-style approach, at each step, we look at all available scalar instructions (whose source operands have all been scheduled), pick the largest set with the same operation, and schedule them together.
This na\"ive strategy makes no guarantees about values being computed and used on the same lane; in other words, lining the computation up on incurs arbitrarily many shuffles.
Unlike in normal vectorization, where applying arbitrary permutations to the lanes is relatively cheap, in FHE we are only allowed to rotate the entire vector by a fixed number of slots, and this rotation operation is expensive.
Hence, the cost of bookkeeping quickly outweighs whatever benefit we might get from vectorization, making this approach not worth it.

When trying to vectorize
%\raghav{Terminology things once again: I need a word for ``vertical alignment'' that isn't ``schedule'', since I want to use ``schedule'' to refer to the combination of vertical {\em and} horizontal alignment\dots} 
an FHE program, we have two optimization problems to solve: instruction scheduling, and data layout.
Optimizing only for instruction scheduling gives us the SLP approach: aggressively pack together isomorphic instructions without worrying about the incurred data movement overhead.
Optimizing for data layout places us on the other end of the spectrum: to avoid having to do any rotates, we must place each connected component of the circuit on a single lane, precluding any vectorization and forcing us to execute everything as scalar operations.

One of our key insights is that these two problems are highly related, so we have to solve these {\em simultaneously} rather than independently, attempting to choose an optimal point in the tradeoff space between the two ends of the spectrum.
In the following sections, we lay out the exact optimization problem as well as how we search for a solution. 

\subsection{Overview}
%\raghav{How's this?}
The input to the compilation process is an arithmetic circuit, represented as a directed acyclic graph (DAG), where each vertex corresponds to an operation (gate) in the circuit and the leaves (vertices with no children) correspond to the inputs, and there is an arc $v_1 \to v_2$ if $v_1$ consumes $v_2$. 
When a particular input is used multiple times in the circuit, it can either be represented as a single vertex with an incoming arc from every gate that consumes it, or it can be {\em replicated} into multiple vertices which each get consumed once.
This choice is expressed by the programmer in the surface language (Section~\ref{sec:surface-language}).

The vector {\em pre-schedule}\footnote{This structure is referred to as a {\em pre-schedule} to distinguish it from the actual vector schedule, which explicitly computes an alignment for sequences of instructions at the same level. %\raghav{does this make sense?}
} is a labeled quotient of the original circuit graph, where each vertex represents a connected subgraph, and is labeled with an integer representing the lane the subgraph gets placed on, such that no two vertices at the same height are labeled by the same lane.
The pre-schedule is naturally bi-graded into {\em epochs}, or groups of (independent) vertices at the same height which get packed together into a single sequence of vector instructions requiring no data movement, as well as {\em columns}, groups of vertices assigned to the same lane representing computation that happens in a single thread with no internal vectorization.
% Note that the uniqueness condition stated above means that each vertex of the quotient graph can be uniquely identified by its epoch and column (although the converse in general does not hold, since its possible to have a column that does not contain every epoch and vice-versa).

It turns out that both of our extremes from earlier can be realized in this model.
Aggressively vectorizing SLP-style can be recovered by assigning a trivial subcircuit to each vertex of the quotient, and simply enumerating lanes across epochs.
On the other end of the spectrum, we could instead quotient the circuit into the discrete graph of its connected components and assign each discrete vertex an arbitrary lane; since this graph has no edges it requires no rotations, but also precludes any vectorization within connected components.

Finding a good pre-schedule then requires us to first compute a ``good'' quotient that trades off between these extremes, together with a lane assignment that somehow maximizes our ability to vectorize without incurring too many rotations.
This is expressed in the search procedure {\system} uses when finding a vector schedule: an outer loop performs a best-first search over possible quotient graphs, and an inner loop uses simulated annealing on each quotient to find a good lane placement.
The result of the search procedure is a quotient of the circuit and a lane placement for the quotient, which together minimize\footnote{relative to the other quotients and lane placements visited in the search} the cost of the resulting vector schedule. %\raghav{how is this (including the footnote)?}
The next section discusses this search procedure in more detail.
% The next section discusses what makes one graph quotient ``better'' than another, and exactly how these tradeoffs are quantified.

\subsection{Schedule Search}\label{sec:schedule-search}
Given a cost model, we use a two-layer optimization strategy to produce a schedule that has good packing properties without incurring too much data movement overhead.
\subsubsection*{Determining lane placement}
\begin{algorithm}[t]
    \SetKwFunction{placelanes}{PlaceLanes}
    \SetKwFunction{naive}{InitialPlacement}
    \SetKwFunction{permute}{GenerateCandidate}
    \SetKwFunction{cost}{Cost}
    \SetKwFunction{accept}{Accept}
    
    \SetKwProg{algo}{Algorithm}{}{}
    \SetKwProg{proc}{Procedure}{}{}

    \algo{\placelanes{graph}}{
        $lanes \gets \naive{graph}$\;
        $T \gets T_0$\;
        $cost \gets \cost{lanes, graph}$\;
        \For{$i = 1 : N$}{
            $T \gets T / (1 + \beta T)$\;
            $candidate \gets \permute{lanes, graph}$\;
            $cost' \gets \cost{candidate, graph}$\;
            \If{$\accept{cost, cost', T}$}{
                $lanes\gets candidate$\;
                $cost \gets cost'$\;
            }
        }
        \Return{lanes, cost}
    }
    \proc{\cost{lanes, graph}}{
        $rotations[*] \gets \emptyset$\;
        \ForEach{$(u\to v) \in graph$}{
            \If{$lanes[u]\neq lanes[v]$}{
                $rotations[u.epoch] \gets rotations[u.epoch]\cup \{lanes[v] - lanes[u]\}$\;
            }
        }
        $instrs[*]\gets 0$\;
        \ForEach{$epoch \in graph$}{
            \ForEach{$opcode$}{
                $instr[opcode] \gets instr[opcode] + \max\limits_{col}count(epoch, col, opcode)$\;
            }
        }
        \Return{$w_R\times\sum\limits_{ep} rotations[ep] + \sum\limits_{op} w_{op}\times instrs[op]$}
    }
    \caption{Lane placement}\label{alg:lane-placement}
    
\end{algorithm}
The inner layer uses simulated annealing to find an optimal lane assignment for a given quotient graph.
The initial assignment is the naive one given by simply enumerating the vertices at each epoch.
At each step of the algorithm, we generate a candidate solution by randomly choosing two columns and a subset of the epochs in them to swap, maintaining the uniqueness condition of the schedule.
If the overall cost (as described in Section~\ref{sec:cost-model})
% \footnote{We could have chosen to only look at the rotation cost of each solution, but considering both costs helps prioritize lane placements with a heterogeneity of instruction types within each lane} 
of the candidate solution is lower than the original cost, it is accepted, and used as the starting point for the next round.
If the candidate solution cost is {\em higher} than the original cost, it is accepted with a probability that varies negatively with the difference in cost, and is generally smaller in later rounds than in earlier rounds\footnote{We use a slow cooling schedule with initial temperature $T_0=50$ and cooling parameter $\beta=10^{-3}$. The probability of accepting a move that increases the cost by $\Delta_c$ is $e^{-\Delta_c/T}$. The annealing is run for $20$k rounds.}.
After a fixed number of rounds have elapsed (see footnote), this algorithm returns the best solution found so far.

\subsubsection*{Computing optimal circuit quotient}
\begin{algorithm}[t]
    \SetKwFunction{quotient}{ComputeQuotient}
    \SetKwFunction{placelanes}{PlaceLanes}
    \SetKwFunction{accept}{Accept}
    \SetKwFunction{contract}{ContractEdge}
    \SetKwFunction{condensation}{Condensation}
    \SetKwFunction{cross}{CrossArcs}
    \SetKwFunction{enq}{Enqueue}
    \SetKwFunction{deq}{Dequeue}

    \SetKwProg{algo}{Algorithm}{}{}

    \algo{\quotient{graph}}{
        $lanes, cost \gets \placelanes{graph}$\;
        $best \gets lanes, graph$\;
        $bestcost \gets cost$\;
        $pqueue\gets []$\;
        $\enq{pqueue, (graph, lanes), cost}$\;
        \For{$i = 1 : N$}{
            $graph, lanes \gets \deq(pqueue)$\;
            \If{$arc \samplefrom \cross{graph}$}{
                $candidate \gets \condensation{\contract{graph, arc}}$\;
                $lanes', cost' \gets \placelanes{candidate}$\;
                $\enq{pqueue, (candidate, lanes'), cost'}$\;
                \If{$cost' < bestcost$}{
                    $best \gets lanes', candidate$\;
                }
                $\enq{pqueue, (graph, lanes), cost}$\;
            }
        }
        \Return{best}
    }
    \caption{Computing a good circuit quotient}\label{alg:circuit-quotient}
\end{algorithm}
The outer layer searches the space of quotients for a graph that admits a good lane placement without giving up too much vectorizability.
Here, we use a priority queue to implement a simple best-first search.
Each graph in the queue is assumed to already be equipped with an optimal lane placement, via the algorithm described above.
At each step, a graph is dequeued, and a new candidate solution is generated by looking at its set of cross-lane arcs and choosing one to contract (removing the edge and identifying its endpoints into a single vertex).
The contracted graph may not be acyclic, so we continue contracting cycles until it is (in effect computing the condensation). %\raghav{better?} %first mod out by all cycles before running the lane placement algorithm again.\milind{It's not clear what ``mod out by all cycles'' means -- add more meat here...}
The candidate solution is then enqueued with a priority based on its cost from the annealed lane placement.
If there are more available arcs to contract, the original graph is enqueued again.

After a fixed number of rounds have elapsed, or once the queue is empty, the algorithm terminates and returns the best graph.
Since each step of this algorithm involves an expensive call to the lane placement procedure, this runs for a much smaller number of rounds, usually between 150 and 200.
In practice, this is enough to find highly efficient schedules.

The next section discusses what makes one graph quotient or lane assigment ``better'' than another, and how these tradeoffs are quantified in \system's cost model.

\subsection{Cost Model}\label{sec:cost-model}
The cost of a particular pre-schedule comes from two places: the number of rotations we have to perform, and the amount we have ``given up'' on vectorizing.
\subsubsection*{Rotations}
Given a vector schedule, each {\em cross-lane arc} in the graph (an arc connecting vertices of different lanes) represents a rotation that must be performed to align an output from the tail of the arc to where it gets used at the head.
However, determining the rotation overhead is not as simple as counting these arcs.
Consider the case where instructions $A$ and $B$ are operands to instructions $A'$ and $B'$, respectively.
If $A$ and $B$ are assigned lanes $n$ and $m$, $A'$ and $B'$ are assigned $n+k$ and $m+k$, and $A$ and $B$ end up packed together in the same vector instruction, the two separate data movement operations required for the $A\to A'$ arc and the $B\to B'$ can actually be performed by a single rotation by $k$ (in fact, taking advantage of this fact is the main way \system optimizes data layout to require fewer total rotates). 
To compute the {\em actual} number of required rotations, we instead proceed epoch-by-epoch. 
For each epoch, we look at all cross-lane arcs with tails in that epoch, and compute the number of columns each spans (i.e. the required rotation amount) by subtracting the lane at the tail from the lane at the head.
The rotation cost for that epoch is then just the number of distinct rotation amounts.
For example, if a particular epoch has five cross-lane arcs, of which three represent a rotation of $-1$ and two represent a rotation of $6$, its rotation cost is $2$.
It follows that the total rotation cost of a schedule is the sum of the rotation costs of each epoch.

\subsubsection*{Vectorizability}
Taking successive quotients of the circuit reduces the total number of edges, and by extension, reduces the number of rotates that might be required; however, it also precludes any vectorization within the collapsed subcircuits.
To account for this, we need a way of quantifying the amount of vectorization we are ``giving up'' with each quotient.

Unfortunately, directly computing the opportunity cost is very messy: the amount of vectorization we give up by identifying a set of vertices is not a property local to the vertices, but rather requires us to look globally at {\em all possible vertices} in those epochs, to see which vectorization opportunities are no longer available after the identification.
Instead, we use an estimated {\em schedule height} as a proxy, with the justification being that giving up a lot of vectorization generally results in taller, less efficient final schedules. 
%\milind{this is good.}\raghav{Is it enough to say this? Or do I need to justify it further? Or should I remove these paragraphs entirely and just directly describe the schedule height computation?}

The schedule height computation once again proceeds epoch-by-epoch.
For each epoch, we estimate the minimum number of vector instructions required after packing by taking the maximum number of each type of operation across all the subcircuits associated to the vertices in that epoch.
For example, the estimated height of an epoch containing one vertex with 3 adds and 2 multiplies and another vertex with 2 adds and 4 multiplies would be 3 adds and 4 multiplies.

\subsubsection*{Overall Cost}
The analysis presented above computes an estimate for the number of each type of instruction in the generated vector program.
The final cost used a linear combination of all of these, with weights determined empirically by how expensive each instruction type is relative to the rest.
In our implementation, we scale rotates and multiplies by $1$, and addition and subtraction by $0.1$. 
%\raghav{Should this last sentence go here, or in implementation? Or both?}\milind{Both is good.}

\subsection{Instruction Alignment}\label{sec:instruction-alignment}
%\raghav{I just lifted this from the old version, is it still good or do we need to update it?}
We align the instructions corresponding to the subcircuits in each epoch to produce a final vector schedule.
% Given a set of subcircuits to compute in a single epoch of the program, we can align the instructions between subcircuits to actually produce a vector schedule.
It may seem like the solution to this is just sequence alignment, but aligning circuits is actually more complicated.
At each step, the number of available children to align roughly doubles, meaning that the total number of subproblems to solve is exponential in the depth instead of linear. 
This causes the dynamic programming strategy of sequence alignment to quickly blow up.

Instead of wrangling so many subproblems, we can formulate this as an ILP.
We create a variable for each scalar instruction representing its {\em schedule slot}, or the time at which it executes.
We add constraints to require that each instruction be scheduled after all of its dependences, and also that two instructions with different operations never be scheduled at the same time. 
Finally, to speed up the search for a solution, we place a bound on the total length of the schedule which is iteratively tightened until the solver returns ``unsatisfiable'', meaning no shorter schedule could be found. % (so the most recent one was the shortest).

\subsection{Data Layout}\label{sec:data-layout}
The circuit obtained after vectorization necessarily operates on inputs that have been ``packed'' into vectors.
Choosing a good layout within these vectors is crucial, since a poor choice of layout could incur many additional rotations to line operands up with where they are used.
{\system} can automatically select a good layout as part of the vectorization process.
An input that is only used once is placed on the lane within its vector corresponding to the unique lane where it is used, and any two inputs that are placed on the same lane by this rule are packed into separate input vectors to avoid collisions.

For inputs that are used multiple times (or inputs that are required to be packed into the same vector, e.g. elements of the same matrix), {\system} places a no-op ``load'' gate in the scalar circuit (so that the input is only used once, by the load gate).
Two load gates are placed in the same epoch in the circuit if and only if their corresponding inputs are required to be packed together (thus ensuring that they are given different lanes).
The layout for these inputs is then determined by the lanes chosen for their corresponding load gates.
This determines the data layout, as each input is placed on the same lane as its corresponding load gate (Section~{\ref{sec:using-coyote}}). %\milind{forward reference to 5.1 for how this can be overriden}
% The layout for these inputs is then determined by the lanes chosen for their corresponding load gates. \raghav{I could explain this better, but I'm not sure how.}

\section{Implementation}\label{sec:implementation}
\subsection{Code Generation}
The output of the algorithm in Section~\ref{sec:design} is a vector schedule (i.e. a lane and schedule slot for each scalar, where the schedule slot determines the order in which instructions get executed).
This schedule is then compiled into the actual vector IR, which supports vector addition, subtraction, multiplication, and rotation instructions, as well as a constant load instruction and a {\em blend} instruction.
A blend instruction does no data movement, but takes several vectors and ``blends'' them together by choosing specific lanes to take from each one.
In practice, this is implemented as a series of plaintext/ciphertext multiplies (where each ciphertext vector is multiplied by a plaintext ``bitmask'' to hide certain lanes), followed by several ciphertext/ciphertext adds, where each of the masked vectors is added together.

Throughout the compilation, we keep a lookup table that maps a (scalar, lane) pair to the name of the vector register containing it.
In particular, there may be several vector registers that contain a particular scalar, but on separate lanes (this results from a rotation of the original register where the instruction was produced).
(For example, the table might say ``The vector register {\texttt s2} has scalar 11 on lane 3'').

At every time step, the schedule compiler finds all scalar instructions scheduled to execute.
For each operand, we use the table to look up the name of the register containing that scalar in the appropriate lane.
If the data for a single vector operand comes from multiple registers, we emit a blend instruction, which multiplies each of the registers by a plaintext mask and adds them, essentially ``blending'' together their values.
Once all the operands have been blended and prepared, we emit the appropriate vector add, subtract, or multiply instruction.
We then look ahead to see if any of the scalars produced by this vector instruction get used on different lanes; for each distinct rotation this induces, we emit a vector rotate instruction.
To standardize, all rotation amounts are positive modulo the vector width.
Finally, for each vector register just produced (including the original one and any rotated ones), we add all of its scalars and their lanes to the lookup table and proceed to the next time step.

Once the circuit is compiled to the vector IR, it can be further lowered to C++ SEAL code.
This is a very straightforward process, and essentially consists of transliterating the IR. 
SEAL does not support a built-in blend instruction, though, so each of these generate several ctxt/ptxt multiplies followed by a single ctxt add.
While lowering to C++, we also precompute all the blending masks, so that we can encode all of them into the plaintext space once at the beginning of the program and just look up the appropriate one to use each time.

% \begin{enumerate}
%     \item Generate arithmetic circuit within python, pass it to the compiler which tags it and produces scalar code
%     \item Take compiler object (mostly contains metadata + scalar code) and pass it to vectorization function, which goes through the algorithm described in section~\ref{sec:design} and uses it to produce a vector schedule (this consists of assigning a lane and a time to each scalar instruction).
%     \item Given a vector schedule, we compile it into the actual vector IR that consists of loads, vector adds/subtracts/multiplies, and rotations.
%     \item The vector IR as well as the original scalar code are both translated into C++ code for SEAL and put into the appropriate place in a pre-configured CMake project.
%     \item Building the CMake project links the generated C++ against our test harness, which runs both the vector and scalar C++ code a specified number of times, collects timing information, and dumps it to a CSV.
% \end{enumerate}

\subsection{Optimality Tradeoffs}
Because each of the compilation steps quickly blows up when given larger programs, we make a number of tradeoffs that sacrifice some optimality in exchange for faster compilation times.
\subsubsection*{Finding maximal cliques/synthesizing alignment}
In many cases, we want a certain optimal solution to the problem we are passing to the solver; for instance, we want the clique with the largest sum of edge weights, or we want the vector schedule that uses the fewest schedule slots.
One way to accomplish this is to pass an objective function to the solver; however, this technique often takes a long time, since before returning a solution the solver must first prove that no better one exists.
Instead of supplying an objective function, we first find any legal solution (e.g. any clique, or any schedule that respects instructions and dependences).
We then add a constraint that require the next solution to be strictly better than the first one, and query the solver again, until eventually it returns ``unsatisfiable'', meaning that no better solution to be found.
We can then set a time limit so that if no better solution is found within it, we use the best one we have so far.
By varying the time limit, we can trade off between optimality and compilation time.

\subsubsection*{Computing graph paths}
To build the hypergraph described in Section~\ref{sec:lane-placement}, we first need to compute all paths through the dependence graph.
This is accomplished by starting with all paths of length one (edges) and inductively computing transitive closures, keeping track of all the cycles we find along the way. \raghav{Did I say that right?}
However, for very large or complicated dependence graphs, this can take a very long time, so we once again set a time limit after which we stop looking for paths (in practice, this amounts to choosing a maximum length of path to look for).
This means that when the time limit is hit, we miss some relations, meaning that even a properly colored hypergraph may produce an unsolvable integer program.
In these cases, the solver produces an ``unsat core'', witnessing the unsatisfiability as a set of paths through the dependence graph that start and end on the same epoch, but sum to the wrong thing (e.g. a cycle that sums to a nonzero value).
When we encounter this, we ``uncolor'' all the edges along such paths, allowing the solver to assign them whatever values it needs to in order to make the program satisfiable.
Doing so breaks certain symmetries (e.g. an uncolored edges may no longer have the same value as another edge, necessitating an extra rotation), but the lane assignment is still correct-by-construction.
Once again, this amounts to sacrificing optimality in exchange for compilation time: simply by increasing the time limit on finding the paths, we can avoid missing relations, and ensure that we never have to do this.

\subsection{Duplicating Inputs}
\section{Evaluation}\label{sec:eval}
\subsection{ML-Adjacent Kernels}
\begin{figure*}
    \begin{subfigure}{0.3\textwidth}
        \includegraphics[width=0.95\textwidth]{figures/graphs/DataUnreplicatedENC+RUN.png}
        \caption{Scalar vs. Vector comparison for unreplicated inputs}\label{fig:ml-kernels-unrepl}
    \end{subfigure}
    \begin{subfigure}{0.3\textwidth}
        \includegraphics[width=0.95\textwidth]{figures/graphs/DataPartiallyReplicatedENC+RUN.png}
        \caption{Scalar vs. Vector comparison for partially replicated inputs}\label{fig:ml-kernels-part-repl}
    \end{subfigure}
    \begin{subfigure}{0.3\textwidth}
        \includegraphics[width=0.95\textwidth]{figures/graphs/DataReplicatedENC+RUN.png}
        \caption{Scalar vs. Vector comparison for fully replicated inputs}\label{fig:ml-kernels-repl}
    \end{subfigure}
    \caption{Scalar vs. Vector encryption + run time comparisons for various replication regimes. The scalar run times (in the red bars) are normalized to 1, so a smaller (blue) vector bar represents more speedup.}\label{fig:ml-kernels}
\end{figure*}

We evaluate \system by compiling several ML-adjacent kernels and comparing their vectorized performance against an unvectorized (scalar) baseline implementation.
Each kernel is used in two benchmarks with differently sized inputs to measure how well they scale with \system's vectorization.
Finally, we compile three versions of each benchmark: once with both inputs replicated, once with only one input replicated, and once with neither input duplicated. 
The exact benchmarks used are:
\begin{itemize}
    \item Matrix/Matrix multiply ($2\times 2$ and $3\times 3$)
    \item Matrix/Matrix multiply followed by determinant ($2\times 2$ and $3\times 3$)
    \item Pairwise distance computation (2 points and 3 points)
    \item Vector dot product (vector size of 3 and 6)
    \item Matrix convolution ($4\times 4$ matrix with $2\times 2$ kernel and $3\times 3$ kernel)
\end{itemize}

Figure~\ref{fig:ml-kernels} shows the performance results for these benchmarks.
The red bars represent scalar execution time (normalized to 1), and the blue bars represent the relative vector execution time (a smaller blue bar is better).
Most of our benchmarks see a greater speedup from vectorization as we move from unreplicated inputs to fully replicated inputs.
This is what we expect, because as discussed in Section~\ref{sec:duplicating-inputs}, replicating a set of inputs leads to fewer rotations necessary to get each of them to the correct lanes.
When fully replicated, all of the vectorized benchmarks are much faster than their scalar counterparts, ranging from a 5$\times$ speedup for \texttt{mat\_mul2x2}, to an approximately $25\%$ speedup for \texttt{mat\_mul\_det3x3}

In the unreplicated runs, we see some of the vectorized kernels (in particular, \texttt{mat\_convol\_4x4x2x2}, \texttt{mat\_mul\_det3x3}, and \texttt{pairwise\_dist2x2}) actually {\em slower} than the scalar baseline.
This is because the overhead of all the rotations these benchmarks incur outweighs any benefits gained from vectorization.
In fact, it makes sense that these benchmarks would behave like that: convolution is a computation in which the values in the kernel get reused over and over, leading to a high number of rotations to move them into place each time.
The $2\times 2$ pairwise distance benchmark is small enough that even the scalar computation doesn't take very long, but the dependence graph is fully connected, essentially leading to the worst case scenario for vectorization.
And computing the determinant at the end of \texttt{mat\_mul\_det3x3} requires a reduction of 9 values, with no symmetries between them to exploit.
However, even in these examples, the vectorized code is no more than 20\% worse than scalar, showing that despite the high rotation costs, \system is still able to properly take advantage of vectorization opportunities.
\raghav{This all kind of seems like a jumble but I hope I'm getting the point across}

\raghav{I also would like to mention somewhere that \system automagically figures out that the best place to split the matmuldet computations is right before the determinant, but I'm not sure where that goes.}

From these results, we see that it is almost always better to fully replicate inputs when vectorizing, unless specifically compiling several composable kernels separately.

\subsection{Fuzzed Microbenchmarks}
\begin{figure}
    \includegraphics[width=0.9\columnwidth]{figures/graphs/TreeGraph.png}
    \caption{Tree graph}
\end{figure}
\section{Related Work}\label{sec:related-work}
There are two main categories of related work. This section first discusses work on FHE vectorization, and then discusses more general approaches to plaintext vectorization \raghav{Your favorite part: finding a better way to taxonomize all of these!}

\subsection{Vectorization in FHE}
Prior work has been done on building vectorizing compilers for FHE applications \cite{CHET, Porcupine}.
CHET \cite{CHET} is a vectorizing compiler for homomorphic tensor programs that automatically selects encryption parameters, and chooses optimal data layout strategies.
CHET is specifically targeted towards optimizing the dense tensor computations found in neural network inference, and does not apply to a broader class of programs, especially those with irregular computations that are not so easily vectorized.
\system makes no assumptions about the domain of the program, and can generalize to vectorizing even highly irregular computations.

Porcupine \cite{Porcupine} is another vectorizing compiler that uses a synthesis-based approach to automatically generate vectorized FHE kernels given a reference implementation.
Porcupine is more general than CHET, but it is still mostly targeted towards regular, easily vectorizable computations.
While Porcupine can, in theory, generate kernels for any computation, its treatment of rotations makes it harder to adapt to vectorizing irregular programs.
Porcupine considers rotations directly as inputs to arithmetic expressions, and relies on a sketch to constrain the search space and a solver to find programs with optimal performance characteristics.
Irregular applications without many apparent symmetries to exploit incur many asymmetric rotations, making it more difficult for the solver to produce an optimal schedule. The search space becomes prohibitively large, and is not tuned towards finding minimal sets of rotations.
In contrast, because \system focuses on irregular programs, it {\em automatically} adds constraints that limit the search space for rotations, and can often find hidden symmetries even in irregular code, enabling it to combine more rotations at once. \raghav{$\uparrow$ I don't know if this sentence is still true.}
% \raghav{Maybe another thing to say (emphasize?) is that the \system and Porcupine approaches are fundamentally different in that the latter relies on a sketch to synthesize good schedules while \system finds them automatically.}

Additionally, Porcupine's synthesis-based approach results in very long compile times for programs with more than a few instructions.
Their solution to this problem is to break large programs into multiple smaller, composable kernels and synthesize solutions to each kernel independently.
This has two major drawbacks: First of all, synthesizing kernels separately means that Porcupine is unable to find optimizations across kernel boundaries.
Secondly, this requires the programmer to manually split their program into many pieces. \raghav{uhh should I flip the order of these?}
\system, on the other hand, can compile much larger programs in a reasonable amount of time, and the circuit quotienting strategy automatically breaks large programs up into smaller sequences of vectorizable instructions. \raghav{TODO: add instruction counts to the eval?}

%\raghav{I think I explained that correctly?}\milind{Sounds good.}

Gazelle \cite{Gazelle} is a framework for secure neural network inference in FHE.
While it is very optimized for a particular use case, Gazelle is not {\em general}: it consists of a library of highly efficient vectorized kernels that useful in neural network applications.
By contrast, \system can take arbitrary kernels and generate efficient vectorized code on the fly. \raghav{Not sure how to put this nicely, actually.}

\raghav{Remove this since I folded EVA into the (ramparts, alchemy, etc.) sentence earlier?}
Encrypted Vector Arithmetic (EVA) \cite{EVA} is another Python DSL that aims to make FHE programming more accessible; in particular, it also supports writing vector computation.
However, unlike \system, EVA requires the program to write their vectorized code directly: it does not support automatically vectorizing arbitrary kernels.
\raghav{Mention that EVA automatically does parameter selection? Or the fact that we could potentially target EVA as a backend if it weren't restricted to CKKS?}

\subsection{Vectorization of Irregular Programs}
Superword-Level Parallelism \cite{SLP} is a technique for automatically vectorizing arbitrary programs.
SLP works by iterating over a sequence of scalar instructions and computing ``vector packs'', or sets of isomorphic instructions that can be packed together into vectors.
Because SLP does not rely on the presence of data-parallel structures like loops to aid in vectorization, it works well even for irregular programs.
When computing vector packs, SLP does not account for how expensive rotations are in FHE, leading to schedules with very high data shuffling costs. 

VeGen \cite{VeGen} is a recent variant of SLP introduces {\em lane level parallelism} that reasons about which lanes are performing which computations, allowing it to reason about rotation costs when building vector packs.
For example, VeGen can reason about the rotation costs to pack together operands for an instruction into a temporary vector, and can use this to decide whether or not packing those instructions is worth it.
However, this reasoning only happens locally, and VeGen does not incorporate information about how instruction packing might affect rotations much later in the program.

\raghav{Spotcheck again?}
goSLP \cite{goSLP} reasons about globally optimal packing, and tries to find lane placements that minimize data shuffling costs. 
However, there are assumptions baked into its cost model that make it fundamentally unsuitable for the FHE setting.
goSLP frames vectorization overhead in terms of the number of {\em pack} and {\em unpack} operations incurred.
For example, permuting the slots of a single vector incurs one unpack, whereas blending the contents of N vectors (without any permutation) incurs N unpacks.
This results in a cost model that implicitly requires wide blends to be much more expensive than arbitrary permutations.
This is incompatible with FHE, where blends are almost free (instantiated as cheap plaintext multiplies and ciphertext adds) whereas a ``bad'' permutation can require O(n) rotates to realize.
In other words, in the FHE world, there are often highly profitable schedules that require many blends and few rotates, but the framing goSLP uses for vectorization overhead will cause it to forego these schedules for more conservative ones.
Additionally, goSLP does lane placement (permutation selection) {\em after} finding vector packs, creating situations like the one we described in Section~\ref{sec:intro} in which the ostensibly profitable packing does not admit a good data layout.
By contrast, \system's cooperative scheduling strategy ensures that this does not happen.

% While goSLP does reason globally about lane placement, it has two drawbacks in an FHE context. First, it packs instructions into vectors before considering shuffling costs, so may over-aggressively vectorize, while \system may forego some vectorization opportunities in the name of getting packing larger expressions without rotation. Second, its lane placement algorithm considers arbitrary permutations, which are impractically expensive in FHE.
% \raghav{goSLP frames vectorization overhead in terms of the number of {\em pack} and {\em unpack} operations incurred. For example, permuting the slots of a single vector incurs one unpack, whereas blending the contents of N vectors (without any permutation) incurs N unpacks. In other words, goSLP bakes in the assumption that wide blends are much more expensive than arbitrary permutations. This is fundamentally incompatible with FHE, since blends are almost free, whereas bad permutations can require several expensive rotations to realize. Essentially, in the FHE world, its often better to produce a schedule that {\em looks much worse} than to produce the traditionally optimal one.}

One class of related work consists of compilers \cite{Ramparts,ALCHEMY, EVA, Cingulata} that automatically lift programs written in a high level DSL into optimized FHE circuits that perform the same computation.
Unlike \system, these circuit optimizations do not include automatic vectorization \footnote{Although \citet{ALCHEMY} and \citet{EVA} do support ciphertext packing, they require the programmer to manually express their computation over vectors.}.

Swizzle Inventor \cite{SwizzleInventor} is a system which automatically synthesizes efficient data movement kernels for vectorized GPU code.
There are two primary obstacles \system faces in directly applying this approach to our setting:
\begin{enumerate}
    \item SwizzleInventor tackles the data movement problem {\em after the kernel has been vectorized}. As we demonstrated earlier\raghav{\footnote{did we?}}, in the world of FHE, packing and data movement are problems that cannot be reasoned about separately, and must be addressed together.
    \item The sketches SwizzleInventor uses to guide its synthesis rely on the ability to efficiently access arbitrary slots of a packed vector. However, this is not possible with FHE vectors without incurring significant rotation overheads.
\end{enumerate}

Diospyros \cite{Diospyros} is an equality saturation-based vectorization strategy that constructs an e-graph of programs that are semantically equivalent to a given specification, and then uses a custom cost model to extract an efficient vector program, together with necessary shuffles.
However, the simplicity of the cost model it associates to various shuffles makes it unsuitable to deal with the peculiarities and inflexibility of FHE rotations.

In \cite{CircuitRewriting}, Lee et al. describe a general method for automatically rewriting arithmetic and boolean FHE circuits according to a cost model by learning semantically sound rewrite rules.
This approach explores the space of scalar rewrites but does not directly deal with vectorization, making it orthogonal to ours: a technique like this could first be applied to an arbitrary computation to transform it into one more amenable to vectorization before applying \system.
%By reasoning globally about the entire dependence graph of the program at once, \system can identify such phenomena when making vector packs.
%\system can also automatically find vector schedules that are amenable to more efficient data movement, and can propagate this information back through the program this to identify more optimal data layouts.

% Talk about SLP
% Talk about FHE (Gentry paper)
% Talk in depth about CHET
% Talk in depth about Porcupine
\section{Conclusion}\label{sec:conclusion}
In this paper, we presented \system, the first vectorizing compiler for arbitrary homomorphic programs that considers FHE's unique cost model for data movement when vectorizing.
\system operates at a coarser level than most vectorizers, allowing it to minimize the data movement overhead of vectorizing by only packing together sufficently similar subexpressions.
By reasoning about the dependence graph of the program globally, \system can also find optimal data layouts that encourage more efficient rotation patterns.
We show that \system can automatically vectorize a large class of useful kernels, showing speedups of up to 6$\times$ over scalar code.

Potential avenues for future work include fine-tuning \system's cost model and adding semantics-preserving transformations that can restructure circuits to be more vectorizable.
\section*{Acknowledgements}
The authors appreciate the feedback from anonymous reviewers from PLDI 2022, OOPSLA 2022 and ASPLOS 2023 that have improved the paper.
This work was partially supported by NSF grants CCF-1908504 and CCF-1919197, as well as Cisco.
\pagebreak
\appendix
\section{Artifact Appendix}

%%%%%%%%%%%%%%%%%%%%%%%%%%%%%%%%%%%%%%%%%%%%%%%%%%%%%%%%%%%%%%%%%%%%%
\subsection{Abstract}

The artifact contains everything necessary to replicate the results of this paper, including:
\begin{itemize}
    \item An implementation of the compiler described in the paper
    \item A backend test harness for profiling the vectorized code Coyote generates
    \item Implementations of all the benchmarks used in the evaluation
    \item Various scripts necessary to automate the process of compiling, running, and collecting data from the benchmarks.
\end{itemize}

Note that there are two experiments omitted from the artifact, as they require nontrivial manual effort to set up and run. These are the {\tt mm.16.blocked} benchmark described in Section~\ref{sec:scalability}, and Figure~\ref{fig:schedule-cost}

\subsection{Artifact check-list (meta-information)}

{\small
\begin{itemize}
  \item {\bf Compilation: } Translates a python program into an arithmetic circuit, vectorizes it, and generates C++ FHE code
  \item {\bf Transformations: } Loop unrolling, function inlining, circuit vectorization
  \item {\bf Experiments: } Compiling real-world benchmarks, compiling randomly generated polynomial programs, experimenting with data layouts
  \item {\bf How much disk space required (approximately)?: } 200MB
  \item {\bf How much time is needed to complete experiments (approximately)?: } 45 minutes - 1 hour for the small version, up to several hours for running all the benchmarks
  \item {\bf Publicly available?: } Yes
  \item {\bf Code licenses (if publicly available)?: } MIT
  \item {\bf Archived: } 10.5281/zenodo.7591603
  
\end{itemize}
}

%%%%%%%%%%%%%%%%%%%%%%%%%%%%%%%%%%%%%%%%%%%%%%%%%%%%%%%%%%%%%%%%%%%%%
\subsection{Description}

\subsubsection{How to access}

The \system compiler can be accessed at https://github.com/raghav198/coyote

\subsubsection{Hardware dependencies}
No specialized hardware is required to use Coyote, beyond whatever may be necessary to efficiently run z3 and SEAL.

\subsubsection{Software dependencies}
\begin{itemize}
    \item The {\tt coyote} compiler is implemented in Python 3.10 and uses the networkx and z3 modules for its analysis
    \item The test harness backend is written in C++ and uses version 3.7 of the Microsoft SEAL library for its FHE implementation
    \item The test harness uses cmake for its build system
\end{itemize}

%%%%%%%%%%%%%%%%%%%%%%%%%%%%%%%%%%%%%%%%%%%%%%%%%%%%%%%%%%%%%%%%%%%%%
\subsection{Installation}

The Dockerfile provided with the artifact automatically builds and installs all dependencies of Coyote. To build and run the Docker image, run the following commands from the directory containing the Dockerfile:

\begin{minted}{Bash}
    $ docker build -t coyote .
    $ docker run -it coyote bash
\end{minted}

%%%%%%%%%%%%%%%%%%%%%%%%%%%%%%%%%%%%%%%%%%%%%%%%%%%%%%%%%%%%%%%%%%%%%
\subsection{Experiment workflow}
This section describes a workflow to reproduce a subset of the results in the paper.
We've recorded the approximate time it takes to complete each step inside the Docker image on a 2020 M1 MacBook Air.
Lets start by building all the {\tt small} benchmarks (all replication sorts for {\tt conv.4.2}, {\tt mm.2}, {\tt dot.3}, {\tt dot.6}, and {\tt dot.10}, as well as the ungrouped {\tt sort[3]}). 
% 13 minutes
\begin{minted}{Bash}
    $ python3 compile_benchmarks.py --preset small    
\end{minted}
We can also build the data layout case study from Section~\ref{sec:data-layout} of the paper, although note that these circuits are considerably larger, so compiling them will take some time:
% 15 minutes
\begin{minted}{Bash}
    $ python3 compile_benchmarks.py --preset layout    
\end{minted}
Lets also build some of the polynomial trees; in particular, we'll build two of the depth 5 trees in each of the three regimes.
% 5 minutes
\begin{minted}{Bash}
    $ python3 polynomial_benchmarks.py -d 5 -r \
    "100-100" "100-50" "50-50" -i 2    
\end{minted}

We can see, for example, some of the Coyote vector IR:
\begin{minted}{Bash}
    $ cat sort_3/vec    
\end{minted}
and the corresponding generated vector C++ code:
\begin{minted}{Bash}
    $ cat bfv_backend/coyote_out/sort_3/vector.cpp    
\end{minted}

To build {\em all} the benchmarks from the paper (small, medium, and large, as well as the layouts and the random polynomials), run the following instead of following the above steps:
\begin{minted}{Bash}
    $ python3 coyote_compile.py benchmarks.py -c "*"
    $ python3 polynomial_benchmarks.py -d 5 10 -r \
        "100-100" "100-50" "50-50" -i 5    
\end{minted}
However, this is not recommended and will take several hours to complete, as several of the circuits being compiled are quite large.

Now, we need to compile all the C++ code and collect data. Although we used 50 runs and 50 iterations in the paper, lets only use 10 of each to make this go faster:
% compile: % compile+run: 13 minutes
\begin{minted}{Bash}
    $ python3 build_and_run_all.py --runs 10 --iters 10    
\end{minted}
You should see some CMake output followed by the encryption and run times for both scalar and vector versions of each circuit. Note that this script will not re-run benchmarks that already have corresponding CSV files in {\tt bfv\_backend/csvs/}.
Once this is finished running, we can look at one of the generated CSV files:
\begin{minted}{Bash}
    $ cat bfv_backend/csvs/sort/sort_3.csv    
\end{minted}
Now that we've collected all the data for these benchmarks, we can generate the graphs:
\begin{minted}{Bash}
    $ python3 figures.py
\end{minted}
This will generate three plots: vector\_speedups.png, case\_study.png, and trees.png. To view these, either attach to the running Docker container (e.g. using VS Code), or copy the files to your host machine:
\begin{minted}{Bash}
    $ docker cp $(docker ps -q):/home/artifact/graphs/ .    
\end{minted}

Compiling all the small benchmarks takes about 13 minutes, generating and compiling the random polynomial benchmarks takes about 5 minutes, compiling the data layout case study takes about 15 minutes, and building and running all the benchmarks takes about 15 minutes.


%%%%%%%%%%%%%%%%%%%%%%%%%%%%%%%%%%%%%%%%%%%%%%%%%%%%%%%%%%%%%%%%%%%%%
\subsection{Evaluation and expected results}

After running through the workflow described above, you should have generated three plots, each of which replicates part of the experiments in this paper:
\begin{itemize}
    \item {\tt vector\_speedups.png} corresponds to Figure~\ref{fig:vector-speedups}
    \item {\tt trees.png} corresponds to Figure~\ref{fig:polynomial-speedups}
    \item {\tt case\_study.png} corresponds to Figure~\ref{fig:data-layout-case-study}
\end{itemize}
Note that the generated graphs may not contain all the experiments found in the paper (for example, not all the benchmarks in Figure~\ref{fig:vector-speedups} are built in the above workflow, and neither are the depth 10 trees in Figure~\ref{fig:polynomial-speedups}, as these take a long time to compile).
However, the speedups should resemble those in the corresponding figures.

%%%%%%%%%%%%%%%%%%%%%%%%%%%%%%%%%%%%%%%%%%%%%%%%%%%%%%%%%%%%%%%%%%%%%
\subsection{Experiment customization}
\subsubsection{Writing a Coyote program}
Coyote is a DSL embedded in Python, so Coyote programs are just Python functions. To tag a function as a circuit for Coyote to compile, first get an instance of the Coyote compiler:
\begin{minted}{Python}
    from coyote import *
    coyote = coyote_compiler()
\end{minted}
Next, use the compiler to annotate your function with input types:
\begin{minted}{Python}
@coyote.define_circuit(A=matrix(3, 3), B=matrix(3, 3))
def matrix_multiply(A, B):
    ...
\end{minted}
For a full discussion of the available types and their compile-time semantics, see Section~\ref{sec:surface-language}
Finally, use the build script {\tt coyote\_compile.py} to invoke the Coyote compiler on the Python file in which this code is saved:
\begin{minted}{Bash}
    python3 coyote_compile.py circuits.py -c \
        matrix_multiply
\end{minted}
\subsubsection{Invoking the Compiler}
The Coyote compiler can be invoked from the command line via {\tt coyote\_compile.py}. The example invocation above does the following:
\begin{enumerate}
    \item It parses {\tt circuits.py} and loads a list of all circuits defined in that file
    \item It uses Coyote to compile/vectorize the specified {\tt matrix\_multiply} circuit
    \item It creates a directory called {\tt matrix\_multiply} and saves intermediate scalar and vector code into that directory
    \item It lowers the intermediate code into C++ and saves it in {\tt bfv\_backend}
\end{enumerate}
The script expects the name of a Python file that defines one or more circuits (as described above), and then takes a number of command-line parameters:
\begin{itemize}
    \item {\tt -l}, {\tt --list}: Lists all the circuits defined in the file and exit, does not actually compile anything
    \item {\tt -c}, {\tt --circuits}: Load the specified circuits from the file and compile them into C++
    \item {\tt -o}, {\tt --output-dir}: Specify the directory into which to place the generated intermediate code (defaults to the directory from which coyote\_compile.py is invoked)
    \item {\tt --backend-dir}: Specify the directory containing the test harness backend (defaults to bfv\_backend/)
    \item {\tt --no-cpp}: Stops after generated the intermediate code and doesn't generate C++
    \item {\tt --just-cpp}: Uses pregenerated intermediate code to generate C++ instead of recompiling the circuit; this fails if it can't find the intermediate code under [output-dir]/[circuit-name]/
\end{itemize}
\subsubsection{Running the test harness}
The backend test harness comes with a CMake file that automatically builds binaries for everything under {\tt coyote\_out/}. The generated binaries perform a number of runs, where each run consists of executing the scalar and vectorized circuits on random encrypted inputs for a number of iterations and then outputting the total time each version (scalar and vector) took to encrypt, as well as run. All these outputs are then saved into a csv file with the same name as the circuit (e.g. running the binary generated from the example above would create a file called {\tt matrix\_multiply.csv}).

The number of runs and iterations default to 50 each (as these are the values used in this paper), but are configurable via cmake. An example invocation that uses 10 runs with 10 iterations each is as follows:

\begin{minted}{Bash}
    $ mkdir bfv_backend/build
    $ cd bfv_backend/build
    $ cmake .. -DRUNS=10 -DITERATIONS=10
    $ make -j16
\end{minted}
%%%%%%%%%%%%%%%%%%%%%%%%%%%%%%%%%%%%%%%%%%%%%%%%%%%%%%%%%%%%%%%%%%%%%
\subsection{Methodology}

Submission, reviewing and badging methodology:

\begin{itemize}
  \item \url{https://www.acm.org/publications/policies/artifact-review-badging}
  \item \url{http://cTuning.org/ae/submission-20201122.html}
  \item \url{http://cTuning.org/ae/reviewing-20201122.html}
\end{itemize}

% \pagebreak
% \clearpage
% \bibliographystyle{ACM-Reference-Format}
\bibliographystyle{plain}

\bibliography{papers.bib}
\end{document}