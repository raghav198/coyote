\section{Conclusion}\label{sec:conclusion}
In this paper, we presented \system, the first vectorizing compiler for arbitrary homomorphic programs that considers FHE's unique cost model for data movement when vectorizing.
\system operates at a coarser level than most vectorizers, allowing it to minimize the data movement overhead of vectorizing by only packing together sufficently similar subexpressions.
By reasoning about the dependence graph of the program globally, \system can also find optimal data layouts that encourage more efficient rotation patterns.
We show that \system can automatically vectorize a large class of useful kernels, showing speedups of up to 6$\times$ over scalar code.

Potential avenues for future work include fine-tuning \system's cost model and adding semantics-preserving transformations that can restructure circuits to be more vectorizable.