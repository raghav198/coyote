\section{Evaluation}\label{sec:eval}

% \milind{Need to intro evaluation here: 
% 1. what research questions we're trying to answer
% 2. high level overview of what we're going to evaluate to answer those questions
% 3. what platforms we're using to do the evaluation
% }

\raghav{how's this?}
We would like to know whether \system's vectorization strategy actually leads to significant speedups on realistic applications, as well as what properties of a program affect \system's ability to vectorize it.
To answer these questions, we evaluate \system on a number of benchmarks that cover many real-world applications and represent both regular and irregular computation.
We also evaluate \system on a set of randomly fuzzed irregular microbenchmarks that vary aspects like computation density and homogeneity to investigate the effects of these properties on vectorizability. \raghav{Further explain what that means here? Or wait until the microbenchmarks section?}
All our experiments are run on a 2020 Apple M1 MacBook with 16 GB of memory. \raghav{I don't need to say stuff about the cores etc. since nothing's being multithreaded, right?}

\subsection{Computational Kernels}
\begin{figure*}
    \begin{subfigure}{0.3\textwidth}
        \includegraphics[width=0.95\textwidth]{figures/newAspectRatioGraphs/DataUnreplicatedENC+RUN.png}
        \caption{Scalar vs. Vector comparison for unreplicated inputs}\label{fig:ml-kernels-unrepl}
    \end{subfigure}
    \begin{subfigure}{0.3\textwidth}
        \includegraphics[width=0.95\textwidth]{figures/newAspectRatioGraphs/DataPartiallyReplicatedENC+RUN.png}
        \caption{Scalar vs. Vector comparison for partially replicated inputs}\label{fig:ml-kernels-part-repl}
    \end{subfigure}
    \begin{subfigure}{0.3\textwidth}
        \includegraphics[width=0.95\textwidth]{figures/newAspectRatioGraphs/DataReplicatedENC+RUN.png}
        \caption{Scalar vs. Vector comparison for fully replicated inputs}\label{fig:ml-kernels-repl}
    \end{subfigure}
    \caption{Scalar vs. Vector encryption + run time comparisons for various replication regimes. The scalar run times (in the red bars) are normalized to 1, so a smaller (blue) vector bar represents more speedup.}\label{fig:ml-kernels}
\end{figure*}

We evaluate \system by compiling several computational kernels, of the sort that might be found in machine-learning code, and comparing their vectorized performance against an unvectorized (scalar) baseline implementation. Both the scalar and vector versions use the same FHE backend.

Each kernel is used in two benchmarks with differently sized inputs to measure how well they scale with \system's vectorization.
Finally, we compile three versions of each benchmark: once with both inputs replicated, once with only one input replicated, and once with neither input duplicated. 
The exact benchmarks used are:
\begin{itemize}
    \item Matrix/Matrix multiply ($2 \times 2$ and $3 \times 3$)
    \item Matrix/Matrix multiply followed by determinant ($2 \times 2$ and $3 \times 3$)
    \item Pairwise distance computation (2 points and 3 points)
    \item Vector dot product (vector size of 3 and 6)
    \item Matrix convolution ($4 \times 4$ matrix with $2 \times 2$ kernel and $3 \times 3$ kernel)
\end{itemize}

Figure~\ref{fig:ml-kernels} shows the performance results for these benchmarks.
The red bars represent scalar execution time (normalized to 1), and the blue bars represent the relative vector execution time (a smaller blue bar is better).
Vector execution ranges from 0.77 to 6.2 times faster than scalar execution.
Almost all benchmarks have substantially faster execution times with vectorization, except those with very dense dependence graphs (such as pairwise distance) or a lot of input reuse (such as matrix convolution), both of which contribute to requiring a lot of data shuffling.

% \milind{Start with the punchline -- don't keep people in suspense: vector execution ranges from $0.xxx$ to $y$ times faster than scalar execution. Almost all benchmarks, except those with ... have substantially faster vector than scalar runtimes, and performing full replication, which means less rotation is needed, is uniformly faster -- up to 5 times faster. Once you give them the punchline, then you can explain the rest of the details.}

% Most of our benchmarks see a greater speedup from vectorization as we move from unreplicated inputs to fully replicated inputs.
% This is what we expect, because, as discussed in Section~\ref{sec:duplicating-inputs}, replicating a set of inputs leads to fewer rotations necessary to get each of them to the correct lanes.
Performing full replication ameliorates both of these, as discussed in Section~\ref{sec:duplicating-inputs}, and is uniformly faster.
When fully replicated, all of the vectorized benchmarks are much faster than their scalar counterparts, ranging from a 6.2$\times$ speedup for \texttt{mat\_mul2x2}, to an approximately $25\%$ speedup for \texttt{mat\_mul\_det3x3}

In the unreplicated runs, we see some of the vectorized kernels (\texttt{mat\_convol\_4x4x2x2}, \texttt{mat\_mul\_det3x3}, and \texttt{pairwise\_dist2x2}) actually {\em slower} than the scalar baseline.
This is because the overhead of all the rotations these benchmarks incur outweighs any benefits gained from vectorization.
In fact, it makes sense that these benchmarks would behave like that: convolution is a computation in which the values in the kernel get reused over and over, leading to a high number of rotations to move them into place each time.
The $2\times 2$ pairwise distance benchmark is small enough that even the scalar computation doesn't take very long, but the dependence graph is fully connected, essentially leading to the worst case scenario for vectorization.
And computing the determinant at the end of \texttt{mat\_mul\_det3x3} requires a reduction of 9 values, with no symmetries between them to exploit.
However, even in these examples, the vectorized code is no more than 20\% worse than scalar, showing that despite the high rotation costs, \system is still able to properly take advantage of vectorization opportunities.
\raghav{This all kind of seems like a jumble but I hope I'm getting the point across}

% \raghav{I also would like to mention somewhere that \system automagically figures out that the best place to split the matmuldet computations is right before the determinant, but I'm not sure where that goes.}\milind{see note below}

From these results, we see that it is almost always better to fully replicate inputs when vectorizing, unless specifically compiling several composable kernels separately.

Visually inspecting the vector code generated by \system, we see that it often automatically finds what are known to be the optimal schedules.
In the dot product kernels, \system first packs all of the multiplications into a single vector operation, and then logarithmically reduces this down into the sum of all the entries. \raghav{Are these actual words?}
In the \texttt{mat\_mul\_det} kernels, \system first identifies the the highly vectorizable matrix multiply part of the computation, and then essentially computes the irregular determinant on a single lane.

% \milind{Add a discussion here about visually inspecting the generated code, and point out where and whether coyote does especially cool stuff, or where it breaks down.}

\subsubsection*{Ideal Speedups}
In addition to collecting data about execution time, we also compute the ideal speedup of each benchmark (i.e. how much faster vectorization would be if data shuffling were free).
We do this by counting the instructions in both the fully replicated vector schedule and the baseline scalar code, treating a ciphertext multiply as being roughly 10$\times$ as expensive\footnote{This is also the ratio \system uses in its internal cost model.} as a ciphertext add or plaintext multiply, and treating each rotation as a no-op.
This data is shown in Table~\ref{tab:ideal-speedup}.
The most vectorizable benchmarks are \texttt{mat\_mul2x2} and \texttt{dot\_product6x6}, with ideal speedups of 7.6 and 5.0 respectively, and the least vectorizable benchmark is \texttt{pairwise\_dist2x2}, with an ideal speedup of approximately 1.3.
This tracks with what we expect, given that the first two are very regular computations and the pairwise distance benchmark represents a worst case scenario for vectorization, being highly irregular with a dense dependence structure.
However, all of these ideal speedups are very close to the actual speedup in execution time as shown in Figure~\ref{fig:ml-kernels-repl}, suggesting that \system is capable of finding highly optimized schedules even in the presence of high data movement costs.
% \milind{Remove the scalar and vector columns from the table, and add an ``observed speedup'' column to the table so people can see that we get pretty close to the ideal speedup}


\begin{table}
    \begin{tabular}{lcccc}
        \toprule    
         & Ideal & Observed\\
        Benchmark & Speedup & Speedup\\\midrule
        \texttt{mat\_convol\_4x4x2x2} & 3.225 & 1.560\\
        \texttt{mat\_mul\_det2x2} & 4.565 & 3.249\\
        \texttt{mat\_mul3x3} & 4.881 & 4.182\\
        \texttt{pairwise\_dist3x3} & 1.969 & 2.137\\
        \texttt{dot\_product3x3} & 2.667 & 1.999\\
        \texttt{pairwise\_dist2x2} & 1.292 & 1.279\\
        \texttt{dot\_product6x6} & 5.000 & 3.558\\
        \texttt{mat\_convol4x4x3x3} & 4.404 & 2.346\\
        \texttt{mat\_mul2x2} & 7.636 & 6.184\\
        \texttt{mat\_mul\_det3x3} & 1.799 & 1.153\\\bottomrule
    \end{tabular}
    \caption{Speedups from vectorization (ideal speedup by instruction count, and observed speedup by execution time)}\label{tab:ideal-speedup}
\end{table}


\subsection{Fuzzed Microbenchmarks}
To investigate how various aspects of program structure affect \system's ability to vectorize, we randomly generate several expression trees according to three regimes:
\begin{enumerate}
    \item {\em Sparse} -- Many operations have one leaf node input, the tree is not very balanced
    \item {\em Dense, homogeneous} -- The expression tree is both full and complete, and all the operations are isomorphic
    \item {\em Dense, nonhomogeneous} -- The expression tree is both full and complete, each operation has a 50/50 chance of being an add or a multiply.
\end{enumerate}
We generate ten trees for each regime: five with a maximum depth of 3, and five with a maximum depth of 6.
The relative speedups for these trees are shown in Figure~\ref{fig:fuzzed-trees}.
We see that aside from the sparse trees of depth 3, all the rest show speedups when vectorizing, which makes sense since the sparse computations tend to be much more linear and have fewer opportunities for vectorization.
The depth 3 trees show on average less speedup than the depth 6 trees, since they also have less work available to vectorize. \raghav{Is this true? I'm basically trying to say that the scalar is already not bad for a tiny tree so what's the point in vectorizing it}

\begin{figure}
    \includegraphics[width=0.9\columnwidth]{figures/graphs/TreeGraphwithNumbers.png}
    \caption{Vectorization speedups on microbenchmarks. Blue/green bars represent vector time, and red/yellow bars represent (normalized) scalar time, with a smaller blue/green bar being better. Bars are split between time execution time and encryption time, with execution time on the bottom (green and yellow) and encryption time on top (blue and red). The numbers on top of the bars represent the speedup of vector over scalar, with the first one including encryption time and the second one excluding it.}\label{fig:fuzzed-trees}
\end{figure}